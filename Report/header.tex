\usepackage{tikz}
\usetikzlibrary{shapes,arrows,fit,calc,positioning}
\usepackage{setspace}  %% Zur Setzung des Zeilenabstandes
\usepackage[english,ngerman]{babel}     %% Sprachen-Unterstuetzung
\usepackage{calc}      %% ermoeglicht Rechnen mit Laengen und Zaehlern
\usepackage[T1]{fontenc}       %% Unterstutzung von Umlauten etc.
%\usepackage[latin1]{inputenc}  %% 
%% in aktuellem Linux & MacOS X wird standardmaessig UTF8 kodiert!
\usepackage[utf8]{inputenc}    %% Wenn latin1 nicht geht ...
\usepackage{subfigure}

\usepackage{amsmath,amssymb} %% zusaetzliche Mathe-Symbole

\usepackage{lmodern} %% type1-taugliche CM-Schrift als Variante zur
                     %% "normalen" EC-Schrift
%% Paket fuer bibtex-Datenbanken
\usepackage[comma,numbers,sort&compress]{natbib}
\bibliographystyle{plainnat}

\newcommand{\tabheadfont}[1]{\textbf{#1}} %% Tabellenkopf in Fett
\usepackage{booktabs}                      %% Befehle fuer besseres Tabellenlayout
\usepackage{longtable}                     %% umbrechbare Tabellen
\usepackage{array}                         %% zusaetzliche Spaltenoptionen

%% umfangreiche Pakete fuer Symbole wie \micro, \ohm, \degree, \celsius etc.
\usepackage{textcomp,gensymb}

%\usepackage{SIunits} %% Korrektes Setzen von Einheiten
\usepackage{units}   %% Variante fuer Einheiten

%% Hyperlinks im Dokument; muss als eines der letzten Pakete geladen werden
\usepackage[pdfstartview=FitH,      % Oeffnen mit fit width
            breaklinks=true,        % Umbrueche in Links, nur bei pdflatex default
            bookmarksopen=true,     % aufgeklappte Bookmarks
            bookmarksnumbered=true  % Kapitelnummerierung in bookmarks
            ]{hyperref}
\usepackage{lipsum}
%% Weiter benoetigte Pakete: datenumber
%% Falls dieses Paket nicht in der Installation vorhanden ist,
%% kann es von der Seite mit diesem Template heruntergeladen werden
%% und in einem LaTeX bekanntem Verzeichnis installiert werden (notfalls
%% dem Verzeichnis mit der Arbeit).
\addtokomafont{sectioning}{\rmfamily}

\usepackage{graphicx}

\usepackage{mathtools}

\usepackage{color} % for colored Text

%\usepackage{physics}

\usepackage[font={small,it}]{caption}
\usepackage{subcaption}
%\usepackage{pgf}
\setlength{\parindent}{2em}
\setlength{\parskip}{0em}
%\usepackage{titlesec}
%\titlespacing*{\subsection}
%  {0pt}{2\baselineskip}{3\baselineskip}
\usepackage[toc,page]{appendix}
  
\usepackage{braket}

\usepackage{listings}
%%%%%%%%%%%%%%%%%%%New Commmands%%%%%%%%%%%%%%%%%%%%%%%%%%%%%%%%%%%%%%%%%%
%%%%%%%%%%%%%%%%%%%%%%%%%%%%%%%%%%%%%%%%%%%%%%%%%%%%%%%%%%%%%%%%%%%%%%%%%%
\renewcommand{\d}{\mathrm{d}}
\newcommand{\dt}{\frac{\partial}{\partial t}}
\newcommand{\dtau}{\frac{\partial}{\partial \tau}}

\newcommand{\startsquarepar}{%
    \par\begingroup \parfillskip 0pt \relax}
\newcommand{\stopsquarepar}{%
    \par\endgroup}  	

